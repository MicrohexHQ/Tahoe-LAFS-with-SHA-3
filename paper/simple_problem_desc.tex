\documentclass[english,12pt,a4paper]{report}
\usepackage[T1]{fontenc} % In case we want special characters
\usepackage[utf8]{inputenc} % We are all writing in UTF-8

\author{Eirik Haver \and Pål Ruud}
\title{Project assignment - Tahoe-LAFS with SHA-3 candidates}
\date{\today}

\begin{document}

\chapter*{Problem Description}

Tahoe-LAFS is a Free Software/Open Source decentralized data store. It
distributes your filesystem across multiple servers, and even if some of the
servers fail or are taken over by an attacker, the entire filesystem continues
to work correctly and to preserve your privacy and security.

One of the basic security components used in Tahoe-LAFS is the cryptographic
hash function SHA-256.

In the light of the worldwide SHA-3 hash competition, this task is about
making a reproducible, automated benchmark which shows how the performance of
Tahoe-LAFS is affected by the performance of the different SHA-3 candidate hash
functions. Before any testing can be done, Python bindings to the C
implementations of the SHA-3 candidates have to be made, since Tahoe-LAFS is
written in the Python programming language.

\vspace{10mm}
Responsible professor: Danilo Gligoroski danilog@item.ntnu.no

\end{document}
